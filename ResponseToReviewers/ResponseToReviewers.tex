\documentclass[10pt,a4paper]{letter}
\usepackage[utf8]{inputenc}
\usepackage{amsmath}
\usepackage{amsfonts}
\usepackage{amssymb}
\usepackage[left=2cm,right=2cm,top=2cm,bottom=2cm]{geometry}

\begin{document}

\signature{Trever T. Hines}

\begin{letter}{}
\opening{Dear Editor,}

% NOTE THE FOLLOWING ERRORS

% STRAIN COVARIANCE WAS INCORRECT

% THE NOISE MODEL WAS INCORRECT BECAUSE I USED A DIRAC RATHER THAN A
% KRONECKER DELTA.

% WE ADDED MORE POSTERIOR DISPLACEMENT FIGURES


We are grateful for the comments from you and the reviewers. We are
pleased that the comments are generally positive and that most of the
concerns seem to be with the presentation of the material rather than
the material itself. We agree that the previous draft required a
substantial amount of clarification in order to be more accessible to
geophysicists. Consequently, we have revised most of the methods
section (Section 2) with the hopes of improving clarity. Details of
our revisions are given below in our response to the reviewers'
comments.

In the revised draft, we have addressed several issues that we found,
which were not brought up by the reviewers:

1) The strain covariance function, eq. 14 of the previous draft, was
incorrect. That covariance function was only correct for the shear
strain. The normal strains differ by a factor of 2. This error was NOT
in our software implementation of the method, and so our presented
results have not changed.

2) There was an error in eq. 20 of the previous draft. We used a Dirac
delta function where we should have used a Kronecker delta.

3) In the previous draft, Figure 5 was the only figure that showed how
well our model fits the GNSS data. The data consists of seven years of
daily displacements from 94 stations, so it is not possible to show
all the data. However, we are now showing the fit at 3 stations,
rather than 1. These stations were chosen to illustrate a slight
over-smoothing in our fit to the data.

4) In addition to clarifying Section 2, we have significantly
rewritten Sections 3 (application to the Pacific Northwest) and
Section 4 (the discussion) so that they have a clearer focus and
structure. However, the content being presented has not changed.

Each of the reviewers comments are written below and followed by our
response. We hope that we have addressed their concerns sufficiently
in our revision.

%% REVIEWER 1
%%%%%%%%%%%%%%%%%%%%%%%%%%%%%%%%%%%%%%%%%%%%%%%%%%%%%%%%%%%%%%%%%%

\textbf{Reviewer \#1:}\newline

\textbf{Review of the paper : Revealing transient strain in geodetic data
with Gaussian process regression}

\textbf{Overview of the work: The paper focuses on a new non-parametric
Bayesian method to estimate transient strain rate from GNSS data. It
is based on Gaussian Process regression. The method also includes an
outlier removal processing step. This technique is then applied to the
detection of slow slip events in Cascadia. The validity of the
detected transient signals is accomplished via comparison with the
seismic tremor recorded along the Cascadia range.}


\textbf{The article is well-written and your algorithm is very interesting. I
think the use of Gaussian Process Regression is novel and open new
roads in GNSS data analysis. I believe that you spend a lot of efforts
in trying to justify most of the assumptions behind your model,
together with the effects of using different kind of prior. Perhaps,
that overshadows slightly your results. You could consider to
reorganize the article to make it more readable for geophysicists.}

I have taken several steps to make my revised draft more readable for
geophysicists. For details, see my response to the editor's comment
``Many of the reviewers’ comments ... ".

\textbf{I also require that a comparison between your outlier removal model
and one other algorithm could help to support your claims about the
efficiency of your method.}

See my response to the comment ``Also, the paper would be improved if
the proposed outlier detection ..."

\textbf{Otherwise, I have underlined a few clarifications needed about the
stochastic model and some paragraphs within the text.}

\textbf{General Comments:}

\textbf{About the method:}

\textbf{The method is well established. The assumptions to justify the various
hypothesis on the deterministic models of the GNSS time series are
generally well supported (seasonal signal, tectonic rate, offsets).}

\textbf{1- Now, the correlation of the stochastic processes is not completely
clear. When the authors justify the model behind equation 3, wij is a
normally distributed, uncorrelated noise. In statistic, it is equal to
a WGN process. Then, the authors introduce a separate parameter (eta)
which models the temporally correlated noise. This parameter is also
following a Gaussian distribution, with a zero-mean and a covariance
matrix $C_{\eta}$ (i.e. line 10 p5). This approach is not common in
modeling GNSS time series. The standard is based on Williams 2003,
where the author established a stochastic model with a covariance
matrix as sum of identity matrix for the white noise and another
matrix which represents the coloured noise. The coloured noise
covariance matrix is defined differently (see the literature – Bock and
Melgar Reviews of Geophysics 2016, He et al., Journal of geodynamics
2017).}

I have changed my description of the data (eq. 3 in the revised draft)
so that it includes a single term, $\eta$, that encompasses correlated
and uncorrelated noise. I then break $\eta$ down into correlated and
uncorrelated components in Section 3.1 (eq. 24). The covariance matrix
for $\eta$ in eq. 25 is more consistent with the literature.

\textbf{2- First, the author should give a summary of stochastic noise
modeling in GNSS. I also think that a discussion is needed to relate
to previous models. Why using separated stochastic models? Also I
would comment on the possibility that the (low) spatial correlation
between the parameters in your deterministic model (i.e. secular
velocities) could be absorbed in the estimation of the covariance
matrix $C_{\eta}$.}

I believe the reviewer is referring to page 4-5 of the original draft,
where I first introduce $\eta$. I agree that a more thorough
discussion on the noise model is necessary, but I do not think this is
an appropriate place in the text for it. The paper is laid out so that
Section 2 contains a general description of the method for estimating
transient strain rates, and Section 3 contains details that are
specific to the particular application. There is no universally
appropriate noise model, so I decided to save a discussion on the
noise model for Section 3. This is clarified in Section 2 of the
revised draft when I say ``The appropriate noise model may vary
depending on the application, and we hold off on specifying the
covariance matrix, $C_{\eta_i}$, until Section 3.1."

\textbf{Now, I am confused When looking at p.8 Section 4.1, the
description of the noise models should be given as a subsection of
Section 2 p.4. I would also improve the literature review. Several
other models have also been discussed (i.e. Montillet et al., 2014
uses a fractional Brownian motion model; ARMA, ARFIMA, GGM or Band
pass noise – see Het et al., 2017).}

Again, I think that the description of the noise model should remain
in Section 3 because it is specific to the Pacific Northwest
application. 

I have added a few more references in Section 3.1 of the revised draft
to emphasize that GNSS noise modeling is a well-worked area. I refer
the reader to Bock and Melgar 2016 and He et al 2017 for a more
complete review.

\textbf{Outlier detection:}

\textbf{1- P7 : I would give a longer summary in the introduction about all
the efforts in outlier detection in GNSS time series. The first
paragraph p.7 is not enough.}

In Section 2.2 of the revised draft, I have expanded my discussion on
how previous studies have treated outliers in GNSS data.

\textbf{2- Also, the paper would be improved if the proposed outlier detection
method is compared with existing ones such as the Hector software
package (Bos et al., 2013). For example, the comparison between this
method and another one could be done when applying the algorithm in p.
13 (last paragraph below equation 18).}

My outlier detection algorithm is quite similar to what is used in Bos
et al 2013 (and many other papers), in that the outliers are
determined based on the residuals for a best fitting model. The only
difference is that I am fitting the data with a stochastic model
rather than a parametric model consisting of a linear trend and
sinusoids. When using a parametric model, I found that deformation
from slow slip events would sometimes be erroneously identified as an
outlier. The stochastic model used in my outlier detection method is
more flexible and able to describe this transient deformation, and so
only high frequency anomalous observations get identified as outliers.
I have included Figure A1 to demonstrate this.

Of course, a parametric model that includes, for example, B-splines
would be just as effective at detecting outliers without removing
geophysical signal. So I am not suggesting that my outlier detection
method is a significant advancement or even a novel method. Instead, I
am only describing my outlier detection method for the sake of
completeness. The previous version of my paper seems to have put too
much emphasis on the outlier detection method, when the main focus
should remain on the method for estimating transient strain rates. To
help keep my paper focused, I placed the description of the outlier
detection method in the appendix.

\textbf{Results:}

\textbf{The results are generally well explained. However, the use of
different priors and the comparison in the text is sometimes
confusing. Perhaps, a table could summary the different priors and the
main results. It would ease the reading of Section 4.2, 4.3 and then
the discussion in Section 5.}

In the previous draft, I was discussing two different priors: a prior
for the outlier detection algorithm, and a prior used in computing
transient strain rates. I acknowledge that this was convoluted. In the
revised draft, all the details of the outlier detection algorithm are
confined to the appendix. I hope this clarifies any confusion.

Also, I have rewritten most of the results section to be more
structured and hopefully have a clearer focus.

\textbf{Minor issues:}

\textbf{P2 line 30 : “fidelity” replace with “reliability”.}

Done

\textbf{P2 line 30-31 “ Developing and improving upon methods for deriving
secular...area of research.” Need references.}

The introduction has been revised and this sentence has been removed.

\textbf{p.2 line 43 :”too large of an area” ... perhaps “ a very large area”}

Done

\textbf{P5 line 32 : “formal data uncertainty”: can you define it?}

By ``formal data uncertainties" I meant ``the derived uncertainties that
accompany the GNSS displacement solutions". The revised text has been
clarified.

\textbf{P 6 : Need reference to show Equation 15.}

I have added a few more steps in the derivation of eq. 15.

\textbf{P 7 line 54-60: This paragraph can be further enhanced with references
(Bock and Melgar, 2016, Gazeaux et al., 2013).}

I have added a reference to Gazeaux et al. 2013.

\textbf{P 8 : first paragraph. add references on Cascadia range (Aguiar,
Melbourne, Scriver 2009), (Melbourne and Szeliga 2005), (Melbourne and
Webb, 2003, 2002).}

We agree that we need to acknowledge the extensive amount of research
on Cascadia SSEs. In the revised draft, we refer the reader to the
review by Schwartz and Rokosky 2007.

\textbf{P.8 line 20 delete www.unavco.org (only in Acknowledgment)}

Done

\textbf{p. 11 line 45 “common mode error” you need to add some references and
define it properly.}

In Section 3.1 of the revised draft, where we elaborate on the noise
model, we describe common mode error by saying ``Another significant
source of noise in GNSS data is common mode error (e.g., Wdowinski et
al. 1997; Dong et al. 2006), which is noise that is highly spatially
correlated." We believe this definition and these references are
sufficient.

\textbf{p.13 : “Wendland functions have compact support and hence their
corresponding covariance.. sparse”. I would just mention that the
associated covariance matrices are sparse.}

I recognize that it may be a bit redundant to say that the covariance
function is compact and thus the covariance matrices are sparse.
However, Wendland covariance functions are typically described as
``compact" in the literature, and I think it is a key adjective that
should not be dropped out. I have left the wording as is.

\textbf{p.13 line 22 what is an “isotropic Gaussian process”? please add a
definition.}

An Gaussian process is isotropic if its covariance function,
$C(x,x')$, can be written as a function of $||x - x'||_2^2$. An
isotropic Gaussian process is also stationary, meaning that its
statistical properties are invariant to translations. We have replace
``isotropic" with ``stationary" in the text because ``stationary" is a
more appropriate, and perhaps more familiar, term.

\textbf{P.20 : “ Using a compact covariance function ..” Again I would just
say that the covariance matrix is sparse and that s why you have the
mathematical simplification.}

My personal preference is to leave the wording as is.

\textbf{p.20 line 33 “ computational burden” I think that refers to p.20 line
60 “prohibitive when using several years of daily GNSS”. It would
interesting to have some number. How do you quantify the computation
time to process longer and longer time series?}

I agree that numerical values would be useful; however, I am reluctant
to quantify the computational cost of evaluation transient strain
rates. There are many factors that influence the computational cost
(whether sparse matrices are used, the type of sparse matrices that
are used, the linear solver algorithm, etc.) and I am not convinced
that my implementation is the most efficient. So I do not give details
on the computation time, but I note in the discussion that this is an
area for further research.

%% REVIEWER 2
%%%%%%%%%%%%%%%%%%%%%%%%%%%%%%%%%%%%%%%%%%%%%%%%%%%%%%%%%%%%%%%%%%
\textbf{Reviewer \#2:}\newline

\textbf{Comments to the Author(s)}
\textbf{Hines/Hetland GJI Review 08/2017}

\textbf{This is a well written paper and I have just a few minor
clarifications and suggestions below.  One point that might be of
interest to address is would it be possible to “approximate” this
approach so that it can run much faster?  I gather from the text that
the method is intensive for CPU and memory use.}

Our approach is indeed computationally intensive and approximation
methods for GPR do exist. In the discussion section for the revised
draft, we suggest exploring these approximation methods in future
research. 

\textbf{Page 3: Last line: min(t,t’) needs as scaling factor to make
units for covariance e.g., m$^2$/yr}

We added the scale factor $\phi$ to the covariance function for
Brownian motion.

\textbf{Page 4: Stochastic model: It would seem that the assumption of no
covariance between the north and east components is not a very good
one since there is a tectonic framework in which the transients occur
and its unlikely this framework aligns along the Cardinal directions.
Maybe add a statement of impact of the neglect and possibly the idea
of reorienting the “axes” to align with the tectonic framework e.g.,
perpendicular to the subduction zone interface in this case?}

This is a good point that we neglected to mention in the previous
draft. We have added a paragraph to Section 2 that explains the
reasoning and implications for this assumption. The paragraph starts
with ``It should be noted that we have ignored any covariances ...".

\textbf{Page 4/5 Equation (3): Is it worth discussing at this point whether
the estimates of u and n (eta) can be separated without explicit prior
knowledge (which we may not have).  For example, orbit modeling errors
on one satellite (e.g., due to an unknown yaw problem in the
spacecraft) wilt generate a spatially and temporally correlated error
in $d_{ij}$.  How does this (stochastically non-modeled) error not project
in the u estimates?}

At this point in the text, it is assumed that the statistical
properties of u and $\eta$ are known, although not yet specified. So I do
not think this would be an appropriate place to discuss a scenario in
which we do not know u or $\eta$. I also think that it is clear that our
results are contingent on having a good model for u and $\eta$. Asking
whether the strain signal can be resolved without a prior statistical
model for u or $\eta$ is asking whether the signal can be separated from
the noise if one does not know anything about the noise or the signal.
I think it goes without stating that signal can not be extracted from
observations without a model of the noise or an assumption of the
signal sought.

In Section 3.1 and 3.2, we discuss how we choose a model for u and
$\eta$. The key piece of ``prior knowledge" that allows us to discern u
from $\eta$ is that u is negligible for inland stations. This allowed us
to isolate and characterize $\eta$ first, and then we characterized u. If
we allowed $\eta$ to be spatially correlated (which we did not), then we
would have conceivably been able to describe the spatially and
temporally correlated noise from, for example, orbit modeling errors.

\textbf{Page 6: Equation 10: Maybe it is discussed later in the paper but it
is probably worthwhile stating at this point how the inverse is
performed with a block 0 in the matrix?}

The matrix being inverted in eq. 10 and 11 of the revised draft is
still invertible with the block 0, provided that the columns of
$G$ are linearly dependent. This is explained in the revised
draft.

\textbf{Page 6: Equation 12: Maybe some additional explanation is
needed here. How are these partial derivatives formed.  My
understanding the u estimates at this point correspond one-to-one with
the positions and times of the GNSS position determinations.  Did I
miss something?}

The posterior displacements, $\hat{u}$, are actually a spatially and
temporally continuous stochastic process. We can evaluated $\hat{u}$
at any position and time that we may be interested in (not just the
position and times of the data). We are also able to compute the
spatial and temporal derivatives analytically. In Section 2 of the
revised draft, we explain how the derivatives are computed in greater
detail. We hope this clarifies any confusion.

\textbf{Page 8: Partly addresses issue raised about eqn 3.  It
probably should be noted that wether conditions east of -121 deg
longitude are different to the coast so there could be possible
problems with this noise model.}

This is also a good point that we neglected to mention in the first
draft. In Section 3.1 of the revised draft, we acknowledge the
potentially dubious assumption that noise east of -121 is
representative of the entire region. We added the disclaimer saying
``We assume the noise at these inland stations is representative of the
noise at all the stations considered in this study, which is probably
a poor assumption since the inland stations are subject to distinctly
different climatic conditions".

%% Editor Review
%%%%%%%%%%%%%%%%%%%%%%%%%%%%%%%%%%%%%%%%%%%%%%%%%%%%%%%%%%%%%%%%%%

\textbf{Editor:}\newline

\textbf{Editor’s Review of GJI MS 17.0593 ‘‘Revealing transient
strain...’’ by Hines and Hetland This is an novel algorithm applied to
a problem of great current interest. Based on the reviewer’s comments
and my own reading, I am recommending moderate revision. The authors’
response should include a description of their changes, and also a
version of the paper on which these can be seen (just as highlights in
the text).}

\textbf{Many of the reviewers’ comments address places where the paper
could be more clearly written, and my overall comment would be the
same. The current version perhaps spends a little too much time on the
background and not enough on explaining the method and its derivation.
As shown by the references below (these include some the authors
cite), and the Dermanis paper, which I have also attached, strain
estimation from scattered data is an area in which there has been a
lot more prior work than the authors mention. I am including these
references not because the authors need to cite many or even most of
them, but only to make the point that the problem of estimating
strain, and of looking for transients, is a familiar one: for many
readers, myself among them, much more familiar than the statistical
methods applied here. So a slightly fuller discussion of these methods
(perhaps a little more background and some reminders of, eg, what a
hyperparameter is) would be welcome to many readers; and increased
clarity of presentation usually means that the paper will be used
and cited more.}

I appreciate this comment, and I have extensively revised the paper in
order to make it more readable. To list some of the major changes:

  1) I properly define a Gaussian process in Section 2. I elaborate on
     its definition to give it a more tangible meaning.

  2) I give more references, including section numbers for books, to
     make it easier for the reader to get additional information.
   
  3) In section 2 of the revised draft, I give an example of a prior
     for transient displacements. I include a figure showing what the
     covariance function looks like and what a corresponding
     realization looks like.
  
  4) I provide a clearer derivation of the posterior transient
     displacements (eq. 6 and 7 of the revised draft). 
  
  5) In the revised draft, eq. 19 and 20 were added to help the reader
     understand how the transient strain rates are computed.

  6) I provide a clearer derivation of the signal-to-noise ratio.
  
  7) I moved details of the outlier detection method to the appendix
     because it is not essential to the text.

  8) I expanded the explanation of the REML method, which is used to
     constrain u and $\eta$ in Section 3.1 and 3.2.

  9) I attempt to avoid esoteric jargon such as ``hyperparameter". In
     the revised draft, ``hyperparameter" has been replaced with
     ``parameter", which does not degrade the precision of our
     description of the method.
  
  10) Most of the paper was reworded / restructured to improve
     clarity. Despite the fact that very little of the original text
     is in the revised daft, there are almost no differences in the
     content being presented.
 

\textbf{I feel the same way about the section on detecting outliers: again, a
well-worked area. Iterative fitting and outlier detection, as an
overall strategy, ha e been used for a long time. I appreciate that
by imposing spatial and temporal conditions you can identify outliers
not otherwise obvious, but it would help to show an example of this,
rather than the current Figure 4, in which most of the identified
cases are ones that any method could easily find.}

I recognize that outlier detection is a well-worked area. I included
the description of the outlier detection method for completeness, and
not because I thought it was an innovative or superior method. In
order to keep the focus of my paper on my method for computing
transient strain, I move the outlier detection method to the appendix.

In my experimentation with the outlier detection algorithm, it turns
out that the prior spatial covariance has virtually no effect on the
detected outliers. However, the temporal covariance is important. The
prior stochastic model needs to be flexible enough to describe
deformation from, for example, SSEs. This is illustrated in Figure A1
of the revised draft.

\textbf{A few specifics (numbers are page+line):}

\textbf{2+26: possibility not risk: the latter is a technical term in seismic
hazard.}

Done

\textbf{2+47-48: I do not think this description of Shen’s method (which I’ve
used and programmed) is right, since it allows the deformation
gradients to vary in space. I’ve always thought of it as a kind of
adaptive kernel smoother, somewhat like applying 2-D loess; so it is
only dependent on nearby points, and has no assumption of uniformity
anywhere. (Actually this raises a more general question: is the method
given here use local or global support?)}

I have updated my description of Shen's method in the revised draft.

\textbf{3+7: could also}

Done 

\textbf{3+32: redo the sentence; as it stands, you are saying that the SCEC
exercise calculates strain rates.}

The text has been clarified to say ``The method described in Holt 2013
was also tested in the SCEC transient detection exercise. Holt 2013
detects transient signal from an estimate of transient strain rates,
and they estimate transient strain rates using a ..."

\textbf{3+37: a geophysical signal}

Done

\textbf{3+47: function not signal}

Done

\textbf{3+56: by a Gaussian, not with}

This sentence has been removed in the revised draft

\textbf{4+37: displacement (not plural)}

This sentence has been removed in the revised draft

\textbf{6+56: I suspect that equation (15) took some effort to derive; could
some details go into an Appendix?}

I added a clearer derivation in the revised draft.

\textbf{Figure 6: I like that the strain uncertainties can be shown, but is
there a reason not to plot principal strains using arrows, showing
errors by an ellipse around the tips?}

I have put a lot of thought into representing strain, and I recognize
that strain is conventionally represented by showing the principal
strains axes. However, there is no good way of showing the
uncertainties for the principal strain axes. For example, consider a
perfectly dilational strain that is perturbed by a small amount of
noise. The orientation of principal strains will be highly uncertain,
but the magnitudes of the principal strains will be well constrained.
If we were to draw some sort of confidence interval around the tips of
the principal strains, then they would have a crescent shape (not an
ellipse). It is unclear how I would write a program to draw these
confidence intervals.

Instead of showing the principal strains, I am showing the normal
strains as a function of azimuth. For a given azimuth, the uncertainty
in the normal strain is computed by linear propagation of the
uncertainties in the strain. In my opinion, it is much easier to show
the uncertainties in an azimuthal representation of strain.

I did not include this justification in the revised draft because I
think it would be too much of a digression.

\vspace{1cm}
\closing{Thank you for your consideration,}

\end{letter}
\end{document}
