% gjilguid2e.tex
% V2.0 released 1998 December 18
% V2.1 released 2003 October 7 -- Gregor Hutton, updated the web address for the style files.

\documentclass{gji}
\usepackage{timet}

\title[Geophys.\ J.\ Int.: \LaTeXe\ Guide for Authors]
  {Geophysical Journal International:
   \LaTeXe\ style guide for authors}
\author[B.L.N. Kennett]
  {B.L.N. Kennett$^1$\thanks{Pacific Region Office, GJI} \\
  $^1$ Research School of Earth Sciences, Australian National
    University, Canberra ACT \emph{0200}, Australia
  }
\date{Received 1998 December 18; in original form 1998 November 22}
\pagerange{\pageref{firstpage}--\pageref{lastpage}}
\volume{200}
\pubyear{1998}

%\def\LaTeX{L\kern-.36em\raise.3ex\hbox{{\small A}}\kern-.15em
%    T\kern-.1667em\lower.7ex\hbox{E}\kern-.125emX}
%\def\LATeX{L\kern-.36em\raise.3ex\hbox{{\Large A}}\kern-.15em
%    T\kern-.1667em\lower.7ex\hbox{E}\kern-.125emX}
% Authors with AMS fonts and mssymb.tex can comment out the following
% line to get the correct symbol for Geophysical Journal International.
\let\leqslant=\leq

\newtheorem{theorem}{Theorem}[section]

\begin{document}

\label{firstpage}

\maketitle


\begin{summary}
 This guide is for authors who are preparing papers for
 \textit{Geophysical Journal International} using the
\LaTeXe\ document preparation system and the GJI class file.
\end{summary}

\begin{keywords}
 \LaTeXe\ -- class files: \verb"gji.cls"\ -- sample text -- user guide.
\end{keywords}

\section{Introduction}

In addition to the standard submission of hardcopy from authors,
\textit{Geophysical Journal International} accepts machine-readable forms
of  papers in \LaTeX.
The layout design for \textit{Geophysical Journal International} has been
implemented as a \LaTeXe\ class file derived from the MN style file for
Monthly Notices of the Royal Astronomical Society. The GJI classfile is
based on the \verb"ARTICLE" style as discussed in the \LaTeX\ manual
\cite {la}. Commands which differ from the standard \LaTeX\ interface, or
which are provided in addition to the standard interface, are explained
in this guide. This guide is not a substitute for the \LaTeX\ manual
itself. Authors planning to submit their papers in \LaTeX\ are advised to
use \verb"gji.cls" as  early as possible in the creation of their
files.  This guide is modified from that produced by Woollatt et al (1994)
to describe the features of the MN style.

A very accessible guide to the features of \LaTeXe and the differences
from the earlier version is provided by Kopka \& Daly \shortcite{kd}.  This
reference provides in chapter 9 a summary of \LaTeX\ error messages and also
a full description of standard \LaTeX\ commands in Appendix F.


\subsection{The GJI document classes}

The use of \LaTeX\ document classes allows a simple change of class
(or class option) to transform the appearance of your document. The
GJI class file preserves the standard \LaTeX\ interface such that any
document which can be produced using the standard \LaTeX\ \verb"ARTICLE"
class can also be produced with the GJI class. However, the measure (or
width of text) is narrower than the default for \verb"ARTICLE", therefore
line breaks will change and long equations may need re-setting.


\subsection{General style issues}

For general style issues, authors are referred to the `Instructions for
Authors' on the inside back cover of \textit{Geophysical Journal
International}. Authors
who are interested in the details of style are referred to Butcher
\shortcite {bu} and The Chicago Manual \shortcite {cm}. The language of
the journal is British English and spelling should conform to this.

Use should be made of symbolic references (\verb"\ref") in order to
protect against late changes of order, etc.

\subsection{Submission of \LaTeX\ articles to the journal}

Papers should initially be submitted in the usual way to:
The Executive Secretary, Royal Astronomical Society, {\em or\/}
the EGS Editor,  {\em or\/} the DGG editor,  {\em or\/}
the American Editor,  {\em or\/} the Pacific Region Editor,
as set out in the Instructions for Authors on the inside back cover of
each issue of Geophysical Journal International.
Four hard copies should be supplied including figures, normally using the
\verb"[referee]" option, for papers with a  high mathematical content the
\verb"[mreferee]" option is recommended. In each case a separate page of
figure captions is preferred.
One of the copies should  be single-sided,
while the other two should  be weight-reduced, by being either
single-spaced or double-sided.   Copies of figures should
also be supplied. Authors should  ensure that their figures are suitable
(in terms of lettering size, etc.)  for the reductions they intend; they
should not attempt to include their figures inside a \TeX\ or \LaTeX\
file by using \verb"\special" or one of the style files for figure
handling.

Note that articles, or revised versions thereof, may not currently be
submitted by electronic mail. However when the article is accepted for
publication the \LaTeX\ file can be sent to the publisher by \verb"ftp"
together with appropriate forms of figures.  Instructions will be provided
following acceptance.

\section{Using the GJI class file}

If the file \verb"gji.cls" is not already in the appropriate
system directory for \LaTeX\ files, either arrange for it to be
put there, or copy it to your working directory. The class file
and related material, such as this guide, can be accessed via the
journal web-site  at
http://www.blackwellpublishing.com/journals/gji under {\em Author
Guidelines}.

The GJI document class is implemented as a complete document class, {\em
not\/} a document class option. In   order to use the GJI style, replace
\verb"article" by
\verb"gji" in the
\verb"\documentclass" command at the beginning of your document:
\begin{verbatim}
\documentclass{article}
\end{verbatim}
is replaced by
\begin{verbatim}
\documentclass{gji}
\end{verbatim}
In general, the following standard document class options should {\em
not\/} be used with the GJI style:
\begin{enumerate}
  \item \texttt{10pt}, \texttt{11pt}, \texttt{12pt} -- unavailable;
  \item \texttt{twoside} (no associated style file) --
     \texttt{twoside} is the default;
  \item \texttt{fleqn}, \texttt{leqno}, \texttt{titlepage} --
        should not be used (\verb"fleqn" is already incorporated into
        the GJI style);
  \item \texttt{twocolumn} -- is not necessary as it is the default style.
\end{enumerate}

In \LaTeX2e the use of postscript fonts and the inclusion of non-standard
options is carried out through the \verb"\usepackage" command, rather than
as options as in earlier versions.  Thus the Times font can be used for
text by including
\begin{verbatim}
\usepackage{times}
\end{verbatim}
on the line immediately after the \verb"\documentclass". If necessary,
\texttt{ifthen} and \texttt{bezier} can be included as packages.

The GJI class file has been designed to operate with the standard
version of \verb"lfonts.tex" that is distributed as part of \LaTeX
. If you have access to the source file for this guide,
\verb"gjilguid2e.tex", attempt to typeset it.  If you find font
problems you might investigate whether a non-standard version of
\verb"lfonts.tex" has been installed in your system.

\subsection{Additional document class options}\label{classoptions}

The following additional class options are available with the GJI style:
\begin{description}
  \item \texttt{onecolumn} -- to be used \textit{only} when two-column output
        is unable to accommodate long equations;
  \item \texttt{landscape} -- for producing wide figures and tables which
        need to be included in landscape format (i.e.\ sideways) rather
        than portrait (i.e.\ upright). This option is described below.
  \item \texttt{doublespacing} -- this will double-space your
        article by setting \verb"\baselinestretch" to 2.
  \item \texttt{referee} -- 12/20pt text size, single column,
        designed for submission of papers.
  \item \texttt{mreferee} -- 11/17pt text size, single column
        designed for submission of papers with mathematical content.
  \item \texttt{camera} -- designed for use with computer modern fonts to
        produce a closer representation of GJI style for camera
        ready material.
  \item \texttt{galley} -- no running heads, no attempt to align
        the bottom of columns.
\end{description}


\subsection{Landscape pages}

If a table or illustration is too wide to fit the standard measure, it
must be turned, with its caption, through 90 degrees anticlockwise.
Landscape illustrations and/or tables cannot be produced directly
using the GJI style file because \TeX\ itself cannot turn the
page, and not all device drivers provide such a facility.
The following procedure can be used to produce such pages.
\begin{enumerate}
  \item Use the \verb"table*" or \verb"figure*" environments in your
        document to create the space for your table or figure on the
        appropriate page of your document. Include an empty
        caption in this environment to ensure the correct
        numbering of subsequent tables and figures. For instance, the
        following code prints a page with the running head, a message
        half way down and the figure number towards the bottom. If you
        are including a plate, the running headline is different, and you
        need to key in the three lines which are marked with \verb"% **",
        with an appropriate headline.
\begin{verbatim}
% ** \clearpage
% ** \thispagestyle{plate}
% ** \plate{Opposite p.~812, GJI, \textbf{135}}
\begin{figure*}
  \vbox to220mm{\vfil Landscape figure to
                go here. \vfil}
  \caption{}
  \label{landfig}
\end{figure*}
\end{verbatim}
\item Create a separate document with the corresponding document style
      but also with the \verb"landscape" document style option, and
      include the \verb"\pagestyle" command, as follows:
\begin{verbatim}
\documentclass[landscape]{gji}
\pagestyle{empty}
\end{verbatim}
  \item Include your complete tables and illustrations (or space for
        these) with captions using the \verb"table*" and \verb"figure*"
        environments.
  \item Before each float environment, use the
        \verb"\setcounter" command to ensure the correct numbering of
        the caption. For example,
\begin{verbatim}
\setcounter{table}{0}
\begin{table*}
 \begin{minipage}{115mm}
 \caption{Images of global seismic tomography.}
 \label{tab1}
 \begin{tabular}{@{}llllcll}
   :
 \end{tabular}
 \end{minipage}
\end{table*}
\end{verbatim}
The corresponding example for a figure would be:
\begin{verbatim}
\clearpage
\setcounter{figure}{12}
\begin{figure*}
 \vspace{144mm}
 \caption{Travel times for regional model.}
 \label{fig13}
\end{figure*}
\end{verbatim}
\end{enumerate}


\section{Additional facilities}

In addition to all the standard \LaTeX\ design elements, the GJI style
includes the following features.
\begin{enumerate}
  \item Extended commands for specifying a short version of the title and
        author(s) for the running headlines;
  \item A \verb"summary" environment to produce a suitably indented
        Summary
  \item An \verb"abstract" environment which produces the GJI style of
        Summary
  \item A \verb"keywords" environment and a \verb"\nokeywords" command;
  \item Use of the \verb"description" environment for unnumbered lists.
  \item A starred version of the \verb"\caption" command to produce
        captions for continued figures or tables.
 \end{enumerate}
 In general, once you have used the additional \verb"gji.cls" facilities
in your document, do not process it with a standard \LaTeX\ style file.

\subsection{Titles and author's name}

In the GJI style, the title of the article and the author's name (or
authors' names) are used both at the beginning of the article for the
main title and throughout the article as running headlines at the top
of every page. The title is used on odd-numbered pages (rectos) and the
author's name appears on even-numbered pages (versos). Although the
main heading can run to several lines of text, the running headline
must be a single line ($\leqslant 45$ characters). Moreover, the main
heading can also incorporate new line commands (e.g. \verb"\\") but
these are not acceptable in a running headline. To enable you to
specify an alternative short title and an alternative short author's
name, the standard \verb"\title" and \verb"\author" commands have been
extended to take an optional argument to be used as the running
headline. The running headlines for this guide were produced using the
following code:
\begin{verbatim}
\title[Geophys.\ J.\ Int.:
       \LaTeXe\ Guide for Authors]
  {Geophysical Journal International:
   \LaTeXe\ style guide for authors}
\end{verbatim}
and
\begin{verbatim}
\author[B.L.N. Kennett]
   {B.L.N. Kennett$^1$
  \thanks{Pacific Region Office, GJI} \\
  $^{1}$Research School of Earth Sciences,
    Australian National University,
    Canberra ACT \emph{0200}, Australia
  }
\end{verbatim}
The \verb"\thanks" note produces a footnote to the title or author.

\subsection{Key words and Summary}

At the beginning of your article, the title should be generated in the
usual way using the \verb"\maketitle" command. Immediately following
the title you should include a Summary followed by a list of key
words. The summary should be enclosed within an \verb"summary"
environment, followed immediately by the key words enclosed in a
\verb"keywords" environment. For example, the titles for this guide
were produced by the following source:
\begin{verbatim}
\maketitle
\begin{summary}
 This guide is for authors who are preparing
 papers for \textit{Geophysical Journal
 International} using the \LaTeXe\ document
 preparation system and the GJI style file.
\end{summary}
\begin{keywords}
 \LaTeXe\ -- class files: \verb"gji.cls"\ --
 sample text -- user guide.
\end{keywords}

\section{Introduction}
  :
\end{verbatim}
The heading `\textbf{Key words}' is included automatically and the key
words are followed by vertical space. If, for any reason, there are no
key words, you should insert the \verb"\nokeywords" command immediately
after the end of the \verb"summary" or \verb"abstract" environment. This
ensures that the   vertical space after the abstract and/or title is
correct and that any
\verb"thanks" acknowledgments are correctly included at the bottom of
the first column. For example,
\begin{verbatim}
\maketitle
\begin{abstract}
  :
\end{abstract}
\nokeywords

\section{Introduction}
  :
\end{verbatim}

Note that the \verb"summary" and \verb"abstract" environments have the same
effect for the documentclass \verb"gji.cls"

\subsection{Lists}

The GJI style provides numbered lists using the \verb"enumerate"
environment and unnumbered lists using the \verb"description"
environment with an empty label. Bulleted lists are not part of the GJI
style and the \verb"itemize" environment should not be used.

The enumerated list numbers each list item with roman numerals:
\begin{enumerate}
  \item first item
  \item second item
  \item third item
\end{enumerate}
Alternative numbering styles can be achieved by inserting a
redefinition of the number labelling command after the
\verb"\begin{enumerate}". For example, the list
\begin{enumerate}
\renewcommand{\theenumi}{(\arabic{enumi})}
  \item first item
  \item second item
  \item etc\ldots
\end{enumerate}
was produced by:
\begin{verbatim}
\begin{enumerate}
 \renewcommand{\theenumi}{(\arabic{enumi})}
  \item first item
       :
\end{enumerate}
\end{verbatim}
Unnumbered lists are provided using the \verb"description" environment.
For example,
\begin{description}
  \item First unnumbered item which has no label and is indented from
        the left margin.
  \item Second unnumbered item.
  \item Third unnumbered item.
\end{description}
was produced by,
\begin{verbatim}
\begin{description}
 \item First unnumbered item...
 \item Second unnumbered item.
 \item Third unnumbered item.
\end{description}
\end{verbatim}

\subsection{Captions for continued figures and tables}

The \verb"\caption*" command may be used to produce a caption with the
same number as the previous caption (for the corresponding type of
float). For instance, if a very large table does not fit on one page,
it must be split into two floats; the second float should use the
\verb"caption*" command with a suitable caption:
\begin{verbatim}
\begin{table}
 \caption*{-- \textit{continued}}
  \begin{tabular}{@{}lccll}
  :
  \end{tabular}
\end{table}
\end{verbatim}

 \begin{figure}
     \vspace{5.5cm}
     \caption{An example figure in which space has been
              left for the artwork.}
     \label{sample-figure}
  \end{figure}

\section[]{Some guidelines for using\\* standard facilities}

The following notes may help you achieve the best effects with the GJI
style file.

\subsection{Sections}

\LaTeX\ provides five levels of section headings and they are all
defined in the GJI style file:
\begin{description}
  \item \verb"\section"
  \item \verb"\subsection"
  \item \verb"\subsubsection"
  \item \verb"\paragraph"
  \item \verb"\subparagraph"
\end{description}
Section numbers are given for section, subsection, subsubsection
and paragraph headings.  Section headings are automatically converted to
upper case; if you need any other style, see the example in section~\ref{headings}.

If you find your section/subsection (etc.)\ headings are wrapping round,
you must use the \verb"\\*" to end individual lines and include the
optional argument \verb"[]" in the section command. This ensures that
the turnover is flushleft.

\subsection{Illustrations (or figures)}

\begin{figure*}
 \vspace{5.5cm}
     \caption{An example figure spanning two-columns
             in which space has been left for the artwork.}
     \label{twocol-figure}
\end{figure*}

The GJI style will cope with positioning of your illustrations and
you should not use the positional qualifiers on the
\verb"figure" environment which would override these decisions. See
`Instructions for Authors' in {\em Geophysical Journal International\/}
for submission of
artwork. Figure captions should be below the figure itself, therefore
the \verb"\caption" command should appear after the figure or space
left for an illustration. For example, Fig.~\ref{sample-figure} is
produced using the following commands:
\begin{verbatim}
\begin{figure}
 \vspace{5.5cm}
 \caption{An example figure in which space has
          been left for the artwork.}
 \label{sample-figure}
\end{figure}
\end{verbatim}

Where a figure needs to span two-columns the \verb"figure*" environment
should be used as in  Fig.~\ref{twocol-figure} using the following commands
\begin{verbatim}
\begin{figure*}
 \vspace{5.5cm}
   \caption{An example figure spanning two-columns
     in which space has been left for the artwork.}
   \label{twocol-figure}
\end{figure*}
\end{verbatim}

\subsection{Tables}

The GJI style will cope with positioning of your tables and you
should not use the positional qualifiers on the
\verb"table" environment which would override these decisions. Table
captions should be at the top, therefore the \verb"\caption" command
should appear before the body of the table.

The \verb"tabular" environment can be used to produce tables with
single horizontal rules, which are allowed, if desired, at the head and
foot only. This environment has been modified for the GJI style in the
following ways:
\begin{enumerate}
  \item additional vertical space is inserted on either side of a rule;
  \item vertical lines are not produced.
\end{enumerate}
Commands to redefine quantities such as \verb"\arraystretch" should be
omitted. For example, Table~\ref{symbols} is produced using the
following commands.
\begin{table}
 \caption{Seismic velocities at major discontinuities.}
 \label{symbols}
 \begin{tabular}{@{}lcccccc}
  Class & depth & radius
        & $\alpha _{-}$ & $\alpha _{+}$
        & $\beta _{-}$ & $\beta _{+}$ \\
  ICB & 5154 & 1217 & 11.091 & 10.258
        & 3.438 &  0. \\
  CMB & 2889 & 3482 & 8.009 & 13.691
        & 0. & 7.301 \\
 \end{tabular}

 \medskip
 The ICB represents the boundary between the inner and outer cores and
the CMB the boundary between the core and the mantle.  Velocities with
subscript $-$ are evaluated just below the discontinuity and
those with subscript $+$ are evaluated just above the discontinuity.
\end{table}
\begin{verbatim}
\begin{table}
 \caption{Seismic velocities at major
          discontinuities.}
 \label{symbols}
 \begin{tabular}{@{}lcccccc}
  Class & depth & radius
        & $\alpha _{-}$ & $\alpha _{+}$
        & $\beta _{-}$ & $\beta _{+}$ \\
  ICB & 5154 & 1217 & 11.091 & 10.258
        & 3.438 &  0. \\
  CMB & 2889 & 3482 & 8.009 & 13.691
        & 0. & 7.301 \\
 \end{tabular}

 \medskip
 The ICB represents the boundary ...
... evaluated just above the discontinuity.

\end{table}
\end{verbatim}

If you have a table that is to extend over two columns, you need to use
\verb"table*" in a minipage environment, i.e., you can say
\begin{verbatim}
\begin{table*}
\begin{minipage}{80mm}
 \caption{Caption which will wrap round to the
          width of the minipage environment.}
 \begin{tabular}{%
      :
 \end{tabular}
\end{minipage}
\end{table*}
\end{verbatim}
The width of the minipage should more or less be the width of your table,
so you can only guess on a value on the first pass. The value will have to
be adjusted when your article is finally typeset, so don't worry
about making it the exact size.

\subsection{Running headlines}

As described above, the title of the article and the author's name (or
authors' names) are used as running headlines at the top of every page.
The headline on right pages can list up to three names; for more than
three use et~al. The \verb"\pagestyle" and \verb"\thispagestyle"
commands should {\em not\/} be used. Similarly, the commands
\verb"\markright" and \verb"\markboth" should not be necessary.

\subsection{Typesetting mathematics}

\subsubsection{Displayed mathematics}

The GJI style will set displayed mathematics flush with the left margin,
provided that you use the \LaTeX\ standard of open and closed square brackets
as delimiters. The equation
\[
 \sum_{i=1}^p \lambda_i =
{\mathrm{trace}}(\mathbf{S})
\]
was typeset in the GJI style using the commands
\begin{verbatim}
\[
 \sum_{i=1}^p \lambda_i =
{\mathrm{trace}}(\mathbf{S})
\]
\end{verbatim}
This correct positioning should be compared with that for
the following centred equation,
$$ \alpha_{j+1} > \bar{\alpha}+ks_{\alpha} $$
which was (wrongly) typeset using double dollars as follows:
\begin{verbatim}
$$ \alpha_{j+1} > \bar{\alpha}+ks_{\alpha} $$
\end{verbatim}
Note that \verb"\mathrm" will produce a roman character
in math mode.

For numbered equations use the \verb"equation" and \verb"eqnarray"
environments which will give the correct positioning.
If equation numbering by section is required the command
\verb"\eqsecnum" should appear after \verb"begin{document}"
at the head of the file.

\subsubsection{Bold math italic}\label{boldmathitalic}

The class file provides a font \verb"\mitbf" defined as:
\begin{verbatim}
\newcommand{\mitbf}[1]{
  \hbox{\mathversion{bold}$#1$}}
\end{verbatim}
Which can be used as follows, to typset the equation
\begin{equation}
  d(\mitbf{{s_{t_u}}}) = \langle [RM(\mitbf{{x_y}}
  + \mitbf{{s_t}}) - RM(\mitbf{{x_y}})]^2 \rangle
\end{equation}
the input should be
\begin{verbatim}
\begin{equation}
  d(\mitbf{{s_{t_u}}}) = \langle [RM(\mitbf{{x_y}}
  + \mitbf{{s_t}}) - RM(\mitbf{{x_y}})]^2 \rangle
\end{equation}
\end{verbatim}

If you are using version 1 of the New Font Selection Scheme, you may
have some messages in your log file that read something like ``Warning:
Font/shape `cmm/b/it' in size~\hbox{$< \!\! 9 \!\! >$} not available
on input line 649. Warning: Using external font `cmmi9' instead on input
line 649.'' If you have such messages, your system will have substituted
math italic characters where you wanted bold math italic ones: you
are advised to upgrade to version 2.


\subsubsection{Bold Greek}\label{boldgreek}

To get bold Greek you use the same
method as for bold math italic. Thus you can input
\begin{verbatim}
\[ \mitbf{{\alpha_{\mu}}} =
\mitbf{\Theta} \alpha. \]
\end{verbatim}
to typeset the equation
\[ \mitbf{{\alpha_{\mu}}} = \mitbf{\Theta} \alpha . \]


\subsection{Points to note in formatting text}\label{formtext}

A number of text characters require special attention
so that \LaTeX\ can properly format a file.

The following characters must be preceded by a
backslash or \LaTeX\ will interpret them as commands:
\begin{quote}
~~~~~~~~~\$~~~\&~~~\%~~~\#~~~\_~~~\{~~~and~~~\}
\end{quote}
must be typed
\begin{center}
\begin{quote}
~~~~~~\verb"\$"~~~\verb"\&"~~~\verb"\%"~~~\verb"\#"
~~~\verb"\_"~~~\verb"\{"~~~and~~~\verb"\}".
\end{quote}
\end{center}

\LaTeX\ interprets all double quotes as closing quotes.
Therefore quotation marks must be typed as pairs of
opening and closing single quotes, for example,
\texttt{ ``quoted text.''}

Note that \LaTeX\ will not recognize greater than or
less than symbols unless they are typed within math
commands (\verb"$>$" or \verb"$<$").

\subsubsection{Special symbols}

The macros for the special symbols in Tables~\ref{mathmode}
and~\ref{anymode}
have been taken from the Springer Verlag `Astronomy and Astrophysics'
design, with their permission. They are directly compatible and use the
same macro names.
These symbols will work in all text sizes, but are only guaranteed to work
in text and displaystyles. Some of the symbols will not get any smaller
when they are used in sub- or superscripts, and will therefore be
displayed at the wrong size. Don't worry about this as the typesetter
will be able to sort this out.
%
\begin{table*}
\begin{minipage}{106mm}
\caption{Special symbols which can only be used in math mode.}
\label{mathmode}
\begin{tabular}{@{}llllll}
Input & Explanation & Output & Input & Explanation & Output\\
\hline
\verb"\la"     & less or approx       & $\la$     &
  \verb"\ga"     & greater or approx    & $\ga$\\[2pt]
\verb"\getsto" & gets over to         & $\getsto$ &
  \verb"\cor"    & corresponds to       & $\cor$\\[2pt]
\verb"\lid"    & less or equal        & $\lid$    &
  \verb"\gid"    & greater or equal     & $\gid$\\[2pt]
\verb"\sol"    & similar over less    & $\sol$    &
  \verb"\sog"    & similar over greater & $\sog$\\[2pt]
\verb"\lse"    & less over simeq      & $\lse$    &
  \verb"\gse"    & greater over simeq   & $\gse$\\[2pt]
\verb"\grole"  & greater over less    & $\grole$  &
  \verb"\leogr"  & less over greater    & $\leogr$\\[2pt]
\verb"\loa"    & less over approx     & $\loa$    &
  \verb"\goa"    & greater over approx  & $\goa$\\
\hline
\end{tabular}
\end{minipage}
\end{table*}
%
\begin{table*}
\begin{minipage}{115mm}
\caption{Special symbols which don't have to be
used in math mode.}
\label{anymode}
\begin{tabular}{@{}llllll}
Input & Explanation & Output & Input & Explanation & Output\\
\hline
\verb"\sun"      & sun symbol            & $\sun$     &
  \verb"\earth"     & earth symbol         & $\earth$   \\[2pt]
\verb"\degr"     & degree                &$\degr$     &
  \verb"\micron"   & \micron               & \micron    \\[2pt]
\verb"\diameter" & diameter              & \diameter  &
  \verb"\sq"       & square                & \squareforqed\\[2pt]
\verb"\fd"       & fraction of day       & \fd        &
  \verb"\fh"       & fraction of hour      & \fh\\[2pt]
\verb"\fm"       & fraction of minute    & \fm        &
  \verb"\fs"       & fraction of second    & \fs\\[2pt]
\verb"\fdg"      & fraction of degree    & \fdg       &
  \verb"\fp"       & fraction of period    & \fp\\[2pt]
\verb"\farcs"    & fraction of arcsecond & \farcs     &
  \verb"\farcm"    & fraction of arcmin    & \farcm\\[2pt]
\verb"\arcsec"   & arcsecond             & \arcsec    &
  \verb"\arcmin"   & arcminute             & \arcmin\\

\hline
\end{tabular}
\end{minipage}
\end{table*}


The command \verb"\chemical" is provided to set chemical species with
an even level for subscripts (not produced in standard mathematics mode).
Thus \verb"\chemical{Fe_{2}^{2+}Cr_{2}O_{4}}" will produce
\chemical{Fe_{2}^{2+}Cr_{2}O_{4}}.


\subsection{Bibliography}

Two methods are provided for managing citations and
references.   The first approach uses the
\verb"\begin{thebibliography}{}"
and \verb"\end{thebibliography}{}" commands.

The second approach uses a simplified scheme using
\verb"\begin{references}" and \verb"\end{references}" commands.

References to published literature should be quoted in text by author
and date; e.g. Draine (1978) or (Begelman, Blandford \& Rees 1984).
Where more than one reference is cited having the same author(s) and date,
the letters a,b,c, \ldots\ should follow the date; e.g.\ Smith (1988a),
Smith (1988b), etc.
The first time you introduce a three-author paper, you should list all
three authors at the first citation, and thereafter, use et al.

\subsubsection{Biblography References in the text}

References in the text are given by author and date, and, whichever
method is used to produce the bibliography, the references in the text
are done in the same way. Each bibliographical entry has a key, which
is assigned by the author and used to refer to that entry in the text.
There is one form of citation -- \verb"\cite{key}" -- to produce the
author and date, and another form -- \verb"\shortcite{key}" -- which
produces the date only. Thus, Rutherford \& Hawker \shortcite{rh} is
produced by
\begin{verbatim}
Rutherford \& Hawker \shortcite{rh}
\end{verbatim}
while \cite{hi} is produced by
\begin{verbatim}
\cite{hi}
\end{verbatim}

\subsubsection{The bibliography}

The following listing shows some references prepared in the style of
the journal; the code produces the references at the end of this guide.
The following rules apply for the ordering of your references:
\begin{enumerate}
  \item if an author has written several papers, some with other authors,
        the rule is that the single-author papers precede the two-author
        papers, which, in turn, precede the multi-author papers;
  \item within the two-author paper citations, the order is determined
        by the second author's surname, regardless of date;
  \item within the multi-author paper citiations, the order is
        chronological, regardless of author's surnames.
\end{enumerate}
%
\begin{verbatim}
\begin{thebibliography}{}
  \bibitem[\protect\citename{Butcher }1992]{bu}
    Butcher J. 1992. \textit{Copy-editing: The
    Cambridge Handbook}, 3rd edn, Cambridge
    Univ. Press, Cambridge.
  \bibitem[\protect\citename{The Chicago Manual }%
    1982]{cm} \textit{The Chicago Manual of Style},
    Univ. Chicago Press, Chicago, 1982.
  \bibitem[\protect\citename{Chao }1985]{ch}
    Chao, B. F., 1985. Normal mode study of the
    Earth's rigid body motions,
    \textit{Geophys. Res. Lett.}, \textbf{12}, 526-529.
  \bibitem[\protect\citename{Hinderer }1986]{hi}
    Hinderer, J., 1986. Resonance effects of the
    earth's fluid core in earth rotation,
    in \textit{Solved and Unsolved Problems},
    pp. 277-296, ed. Cazenave A., Reidel,
    Dordrecht.
  \bibitem[\protect\citename{Lamport }1986]{la}
    Lamport L., 1986,  \LaTeX: \textit{A Document
    Preparation System}, Addison--Wesley, New York
  \bibitem[\protect\citename{Lindberg }1986]{li}
    Lindberg, C., 1986.  Multiple taper harmonic
    analysis of terrestrial free oscillations,
    \textit{PhD thesis}, University of California.
  \bibitem[\protect\citename{Maupin }1992]{ma}
    Maupin, V., 1992. Modelling of laterally
    trapped surface waves with application to
    Rayleigh waves in the Hawaiian swell,
    \textit{Geophys. J. Int.}, \textbf{110}, 553-570.
  \bibitem[\protect\citename{Rutherford
    \& Hawker }1981]{rh} Rutherford, S. R.
    \& Hawker, K. E.,  1981, Consistent coupled
    mode theory of sound propagation for a
    class of non-separable problems,
    \textit{J. acoust. Soc. Am.}, \textbf{71},
    554-564
\end{thebibliography}
\end{verbatim}
Each entry takes the form
\begin{verbatim}
\bibitem[\protect\citename{Author(s), }%
  Date]{tag} Bibliography entry
\end{verbatim}
where \verb"Author(s)" should be the author names as they are cited in
the text, \verb"Date" is the date to be cited in the text, and
\verb"tag" is the tag that is to be used as an argument for the
\verb"\cite{}" and \verb"\shortcite{}" commands. \verb"Bibliography entry"
should be the material that is to appear in the bibliography,
suitably formatted.

\subsubsection{Simplified References and Citations}

The second approach to referencing is taken with permission from
the American Geophysical Union Latex macros

The reference section is started using a
\verb"\begin{references}" command which will
automatically produce a correctly formatted
``Reference'' head.  Each reference is then
preceded by a \verb"\reference"
command.  It is the author's
responsibility to place bibliographic reference
information in the proper order with correct
punctuation.  After the last reference in your
reference section, type an \verb"\end{references}"
command.

Authors may enter properly formatted citations directly
in the manuscript text and enclose those citations in
\verb"\markcite{}" commands.  This approach
marks all citations in your manuscript, but there
is no interaction between the \verb"\markcite"
commands and the reference section.

To create in-text citations, enclose each citation
within a \verb"\markcite" command.
There are two ways to include in-text citations,
depending on the way you phrase your sentence.
You may either include an entire reference within
brackets \markcite{(Merritt et al., 1996)} or you
may mention the author as part of your sentence and
include only the year in brackets, as in \markcite{Ono (1996)}.

As an example
\begin{verbatim}
\begin{references}
\reference
Azimi, Sh.A., Kalinin, A.Y., Kalinin, V.B.,
\& Pivovarov, B.L., 1968.
Impulse and transient characteristics of media
with linear and quadratic absorption laws,
\textit{Izv. Earth Phys.} (English Transl.),
\textbf{2}, 88--93.
\reference
Dahlen, F.A., \& Smith, M.L., 1975.
The influence of rotation on the free
oscillations of the Earth,
\textit{Phil. Trans. R. Soc. London Ser. A},
\textbf{279}, 143--167.
\reference
Durek, J.J., Ritzwoller, M.H.,
\& Woodhouse, J.H., 1993.
Constraining upper mantle anelasticity
using surface wave amplitude anomalies,
\gji, \textbf{114}, 249--272.
\end{references}
\end{verbatim}
produces the reference list
\begin{references}
 \reference
Azimi, Sh.A., Kalinin, A.Y., Kalinin, V.B.,
\& Pivovarov, B.L., 1968.
Impulse and transient characteristics of media
with linear and  quadratic absorption laws,
\textit{Izv. Earth Phys.} (English Transl.),
\textbf{2}, 88--93.
\reference
Dahlen, F.A., \& Smith, M.L., 1975.
The influence of rotation on the free oscillations of the Earth,
\textit{Phil. Trans. R. Soc. London Ser. A}, \textbf{279}, 143--167.
\reference
Durek, J.J., Ritzwoller, M.H., \& Woodhouse, J.H., 1993.
Constraining upper mantle anelasticity
using surface wave amplitude anomalies,
\gji \textbf{114}, 249--272.
\end{references}

\subsubsection{Common Journals}

The following abbreviations are provided for commonly cited
journals and can be used directly in the bibliography.

In the following table the abbreviation and the form of the
associated entry are presented
\newline
\begin{tabular}{ll}
\verb"\areps" & \areps \\
\verb"\bssa"  & \bssa \\
\verb"\eos"   & \eos  \\
\verb"\eps"   & \eps \\
\verb"\epsl"  & \epsl \\
\verb"\gca"   & \gca \\
\verb"\geo"   & \geo \\
\verb"\geop"  & \geop \\
\verb"\gji"   & \gji \\
\verb"\gjras" & \gjras \\
\verb"\grl"   & \grl \\
\verb"\gsab"  & \gsab \\
\verb"\gs"    & \gs \\
\verb"\jgr"   & \jgr \\
\verb"\jseis" & \jseis \\
\verb"\mnras" & \mnras \\
\verb"\pag"   & \pag \\
\verb"\pepi"  & \pepi \\
\verb"\rg"    & \rg \\
\verb"\tecto" & \tecto \\
\end{tabular}
%
% The following two tables have been moved back in the text to
% improve page layout
%
\begin{table*}
\begin{minipage}{130mm}
\caption{Authors' notes.}
\label{authors}
\begin{tabular}{@{}ll}
\verb"\title[optional short title]{long title}"
                    & short title used in running head\\
\verb"\author[optional short author(s)]{long author(s)}"
                    & short author(s) used in running head\\
\verb"\begin{abstract}...\end{abstract}"& for summary on
titlepage\\
\verb"\begin{summary}...\end{summary}"& for abstract on
titlepage\\
\verb"\begin{keywords}...\end{keywords}"& for keywords on titlepage\\
\verb"\nokeywords"  & if there are no keywords on titlepage\\
\verb"\begin{figure*}...\end{figure*}" & for a double spanning figure in two-column mode\\
\verb"\begin{table*}...\end{table*}" & for a double spanning table in
                                       two-column mode\\
\verb"\caption*"    & for continuation figure captions\\
\verb"\resetfigno" & resets figures numbers after an appendix\\
\verb"[referee]" & documentclass option for 12/20pt, single col,
                   for manuscript submission\\
\verb"[mreferee]" & documentclass option for 11/17pt, single col,
                   for submission of papers with extensive mathematics\\
\end{tabular}
\end{minipage}
\end{table*}
%
\begin{table*}
\begin{minipage}{130mm}
\caption{Editors' notes.}
\label{editors}
\begin{tabular}{@{}lp{270pt}}
\verb"\pagerange{000--000}"& for catchline, note use of en-rule\\
\verb"\pagerange{L00--L00}"& for letters option, used in catchline\\
\verb"\volume{000}" & volume number, for catchline\\
\verb"\pubyear{0000}" & publication year, for catchline\\
\verb"\microfiche{GJI000/0}" & for articles accompanied by microfiche\\
\verb"\journal" & replace the whole catchline at one go\\
\verb"[doublespacing]" & documentclass option for doublespacing\\
\verb"[galley]" & documentclass option for running to galley\\
\verb"[landscape]" & documentclass option for landscape illustrations\\
\verb"[fasttrack]" & documentclass option, for rapid short communications
                   (adds F to folios)\\
\verb"[onecolumn]" & documentclass option for one-column \\
\verb"\bsp"& typesets the final phrase `This paper has been produced
 using the Blackwell Publishing GJI \LaTeX2e\ class file.'\\
\end{tabular}
\end{minipage}
\end{table*}

\subsection{Appendices}

The appendices in this guide were generated by typing:
\begin{verbatim}
\appendix
\section{For authors}
     :
\section{For editors}
\end{verbatim}
You only need to type \verb"\appendix" once. Thereafter, every
\verb"\section" command will generate a new appendix which will be
numbered A, B, etc.

If figure captions are to provided after an appendix the figure number can
be reset to avoid extraneous labelling using the command
\verb"\resetfigno".

\section[]{Example of section heading with\\*
  {\mdseries \textsc{S}\lowercase{\textsc{mall}}
  \textsc{C}\lowercase{\textsc{aps}}},
  \lowercase{lowercase},
  \textit{ italic}, and bold\\* Greek such as
  $\mitbf{{\mu^\kappa}}$}\label{headings}

This can be built up using text commands and the \verb"mitbf"
command introduced above

\begin{verbatim}
\section[]{Example of section heading with\\*
  {\mdseries \textsc{S}\lowercase{\textsc{mall}}
  \textsc{C}\lowercase{\textsc{aps}}},
  \lowercase{lowercase},
  \textit{ italic}, and bold\\* Greek such as
  $\mitbf{{\mu^\kappa}}$}\label{headings}
\end{verbatim}

\subsection{Acknowledgments}
Acknowledgments after the main text and before the appendices can be
included with the
\texttt{acknowledgments} environment, as
\begin{verbatim}
\begin{acknowledgments}
We wish to thank ...
\end{acknowledgments}
\end{verbatim}
There is also a corresponding \texttt{acknowledgment} environment for a
single acknowledgment.

\begin{acknowledgments}
A number of colleagues have helped with suggestions for the
improvement of this material and I would particularly like to thank
Bob Geller, University of Tokyo for his criticisms and corrections.
\end{acknowledgments}

\begin{thebibliography}{}
  \bibitem[\protect\citename{Butcher }1992]{bu}
    Butcher J. 1992. \textit{Copy-editing: The Cambridge
    Handbook}, 3rd edn, Cambridge Univ. Press, Cambridge.
  \bibitem[\protect\citename{The Chicago Manual }%
    1982]{cm} \textit{The Chicago Manual of Style}, Univ.
    Chicago Press, Chicago, 1982.
  \bibitem[\protect\citename{Chao }1985]{ch}
    Chao, B. F., 1985. Normal mode study of the Earth's rigid
     body motions, \textit{Geophys. Res. Lett.}, \textbf{12}, 526-529.
  \bibitem[\protect\citename{Hinderer }1986]{hi}
    Hinderer, J., 1986. Resonance effects of the earth's fluid
    core in earth rotation, in \textit{Solved and Unsolved
    Problems}, pp. 277-296, ed. Cazenave A., Reidel,
    Dordrecht.
  \bibitem[\protect\citename{Kopka \& Daly}1995]{kd}
    Kopka H. \& Daly P.W., 1995, \textit{A guide to} \LaTeX2e,
    Addison--Wesley, New York
  \bibitem[\protect\citename{Lamport }1986]{la}
    Lamport L., 1986,  \LaTeX: \textit{A Document
    Preparation System}, Addison--Wesley, New York
  \bibitem[\protect\citename{Lindberg }1986]{bl}
    Lindberg, C., 1986.  Multiple taper harmonic analysis of
    terrestrial free oscillations, \textit{PhD thesis},
    University of California.
  \bibitem[\protect\citename{Maupin }1992]{ma}
    Maupin, V., 1992. Modelling of laterally trapped surface
    waves with application to Rayleigh waves in the Hawaiian
    swell, \textit{Geophys. J. Int.}. \textbf{110}, 553-570.
  \bibitem[\protect\citename{Rutherford \& Hawker }%
    1981]{rh}  Rutherford, S. R. \& Hawker, K. E., 1981,
    Consistent coupled mode theory of sound propagation for a
    class of non-separable problems,
   \textit{J. acoust. Soc. Am.}, \textbf{71}, 554-564
\end{thebibliography}


\appendix
\section{For authors}

Table~\ref{authors} is a list of design macros which are unique to GJI. The
list displays each macro's name and description.

\section{For editors}

The additional features shown in Table~\ref{editors} may be used for
production purposes.

\bsp % ``This paper has been produced using the Blackwell
     %   Publishing GJI \LaTeXe\ class file.''

\label{lastpage}


\end{document}
