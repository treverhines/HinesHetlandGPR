%%gji_extra_guide.tex

\documentclass[extra,mreferee]{gji}
%\usepackage{timet}

\title[\texttt{gji\_extra.sty} ]  
  {\texttt{gji\_extra} - additional facilities for use with GJI
manuscripts} 
  
\author[]  {}  

\newcommand{\btx}{\textsc{BibTeX}}
\begin{document}

\maketitle

\section{Introduction}
The following set of features are derived from other packages with
slight modifications to make them compatible with the \verb"gji"
documentclass. They can be accessed by including the \verb"extra"
option to the \verb"gji" document class e.g.
\begin{verbatim}
  \documentclass[extra,mreferee]{gji}
\end{verbatim}

\section{Reserved Space and Boxed Figures }

It is possible to reserve space for a figure (or to draw a box of
a specified size) using the \verb!\figbox! command.

The \verb!\figbox! command takes 3 arguments: the
horizontal and vertical sizes of the space that is to be left blank, as well
as the contents of the box.


For example,
\begin{verbatim}
\begin{figure}
\figbox{6cm}{5cm}{Paste orbits here}
\caption{The orbits of some planets}
\end{figure}
\end{verbatim}
\begin{figure*}
\figbox{6cm}{5cm}{Paste orbits here}
\caption{Illustrating the use of the figbox command for a
display of the orbits of some planets}
\end{figure*}
makes a box 4~cm wide and 6~cm high with the caption text below it.
The text `Paste orbits here' is simply a note to the author that is
printed centered in the framed box. This third argument could be left empty,
or could contain commands for importing a computer plot.  

There is also a starred version \verb!\figbox*! that behaves exactly the same
as \verb!\figbox! except that no frame is drawn around the figure. This is
most useful for real figures, whereas the unstarred command is more
appropriate for reserved space for glued-in figures.

It is possible to have \verb!\figbox! and \verb!\figbox*! scale automatically
to the size of its contents, something that is also appropriate when the
contents are an imported graphics file. In this case, leave both the
dimensions empty. As an example, if the orbit plot exists as an encapsulated
PostScript file named \texttt{orbit.eps}, then it could be included directly
with the commands
\begin{verbatim}
\begin{figure}
\figbox*{}{}{%
 \includegraphics[width=4cm]{orbit.eps}}
\caption{The orbits of some planets}
\end{figure}
\end{verbatim}
For this to work, you must have loaded the \texttt{graphicx} package with
\verb!\usepackage! at the beginning of the file, and you must have a
PostScript driver for the output. (There are other packages with
different syntaxes for importing graphics; use the one that you are
most familiar with.)


\section{Alternative texts for one or two columns}
Mathematical formulas often have to be fiddled to fit them into the
narrow confines of a single column in two-column format, whereas they
will fit with no problem in the wide columns of the manuscript mode.
This often results in the author having to massage his formulas when he
changes between manuscript and camera-ready options, and then back again
when he want to print the manuscript once more.

The special command \verb!\iftwocol! allows both versions of the text to
be included in the one document, for automatic selection depending on
whether two-column mode is active or not. Its syntax is\label{iftwocol}
\begin{quote}
\normalsize
\verb!\iftwocol{!{\em yes\/}\verb!}{!{\em no\/}\verb!}!
\end{quote}
where {\em yes\/} is the text that is inserted if two-columns are in
effect, and {\em no\/} the text that is otherwise taken.

This command may be used in other situations, but the main application 
is in the case of mathematics.

\section{Literature citations}
\textit{Geophysical Journal International} uses the
author-year system of literature citation, something that is not
supported by standard \LaTeX. The \verb"gji" documentclass
provides partial support but more comprehensive features are 
available using features from the \verb"egs" package and the
\verb"natbib" module developed by P.W.Daly. 

Since there are two ways of making a citation in the author-year system,
either as Jones et al.\ (1990) or as (Jones et al., 1990), there
are two variations of the original \verb!\cite! command.
Suppose the key for the above reference is \texttt{jones90}, then use
\begin{flushleft}
\verb!\citet{jones90}! for Jones et al.\ (1990)\\
\verb!\citep{jones90}! for (Jones et al., 1990)\\
\verb!\citep[p.~32]{jones90}! for (Jones et al., 1990, p.~32)\\
\verb!\citep[e.g.,][]{jones90}! for (e.g., Jones et al., 1990)\\
\verb!\citep[e.g.,][p.~32]{jones90}! for (e.g., Jones et al., 1990, p.~32).

\end{flushleft}
Note that the use of the optional arguments to add notes within the
brackets of the citation:
a single note behaves as in standard \LaTeX, as a note
\emph{after} the citation; however, with two notes (non-standard), the
first goes \emph{before}, the second \emph{after} it.

Two other citation commands are available:
\begin{flushleft}
\verb!\citeauthor{jones90}! prints Jones et al.\\
\verb!\citeyear{jones90}! prints 1990.
\end{flushleft}

For the above examples to function properly, either the \verb"gji"
bibliography style must be used with \btx, or the
\texttt{thebibliography} environment must be formatted accordingly.

\begin{flushleft}
With \btx\\[1ex]
\verb!\bibliographystyle{!gji\verb!}!\\
\verb!\bibliography{!{\em bib\_file\_name\/}\verb!}!\\[1ex]
Without \btx\\[0.5ex]
\begin{verbatim}
\begin{thebibliography}{}
\bibitem[Jones et al.(1990)]{jones90}
  Jones, J. K., Thomas, P. R. 
  \& Peters, R. F., 1990.
  The best results of fitting curves. 
  \textit{J. Math. Dev.}, \textbf{12}, 1245--1261.
\end{thebibliography}
\end{verbatim}
\end{flushleft}

As an example the References at the end of these notes are
the same set as used in \verb"gjilguid2e.tex"
but now cast in the appropriate bibliography environment.

\section{Balancing columns}
It is possible to make the two columns on the last page to be nearly the
same height. It is only necessary to give the command \verb!\balance!
somewhere within the text of the first column on the last page. Issuing
\verb!\balance! too soon can yield strange results on earlier pages;
calling it too late has no effect.

It is best to print out the last page without \verb!\balance! when the
work is nearly completed. If this page has more than one column of text,
then find some appropriate place within the {\em first\/} column,
preferably between paragraphs, and add \verb!\balance! to the text there.
If there is less than one column of text on the last page, then
\verb!\balance! may be inserted anywhere within the text of that column.

\section{planotables}
The {\tt planotable} environment that facilitates the 
formatting of lengthy tables. Tables longer than 
one page will require the use of {\tt planotable}.

It is possible to create fairly complex tables with 
arbitrary spacing, straddle heads, rules, etc. in \LaTeX.
Authors who need to create complicated tables should 
consult the \LaTeX\ manual [{\it Lamport}, 1985] for 
details.  Most of \LaTeX's {\tt tabular} capabilities 
are applicable to {\tt planotables} as well.

Planotables automatically produce table lines and
vertical table spacing.
In addition, planotables have several capabilities that 
facilitate the formatting of tables.  For instance, 
it is possible to break long planotables across pages, 
and, where tabular tables print at an automatic width,
authors may choose a specific planotable width.

Type \verb"\begin{planotable}{COLS}", 
where {\tt COLS}
sets the justification for each column.
Choose one letter (``l,'' ``c,'' or ``r'') for each 
column, indicating left, center, or right justification.

To set a planotable to a specific width, type 
\verb"\tablewidth{}", with the desired table width between 
the curly braces, after the \verb"\begin{planotable}" 
command.  For instance, to create a single-column table 
type \verb"\tablewidth{20pc}", where ``20pc'' represents 20 picas.

Type planotable titles within the curly braces of
\verb"\tablecaption{}" commands.  Capitalize the first
letter of each word (except for prepositions, conjunctions,
and articles that are three letters or shorter).  Do not
allow table caption lines to hyphenate; use a
\verb"\protect\\" command to break lines where necessary.
The \verb"\tablecaption" command automatically generates
the ``{\bf Table \#.}'' information.  

Type a \verb"\startdata" command, and then type your table
data.  The startdata command formats column headings, 
engages the tabular formatting, and produces the table 
caption.  Data within a table row are separated by {\tt \&} 
(ampersand) characters.  The end of each row is indicated 
with a \verb"\nl" command.  Extra vertical space can be 
inserted between rows with a \verb"\vspace{}" command 
(type the desired amount of space between the curly
braces).

If a planotable is longer than one page, \LaTeX\ will 
automatically break it across pages.  To force a page 
break in a particular place, type a \verb"\tablebreak" 
command.  This command affects the following line of 
table data (not the line it appears with) as shown 
in the sample.tex document.

If a cell contains no data, type a \verb"\nodata" 
command to create a ``no data'' symbol.


\subsection{Table footnotes and table comments}

If your table contains material requiring footnotes, use a
\verb"\tablemark{TAG}" command to indicate the footnote,
then type the associated information (with a corresponding
{\tt TAG}) within a \verb"\tablenotetext{TAG}{}" command.

Short table comments can be created using a\\
\verb"\tablenotetext{\null}{TEXT}"\\
command, with a 
\verb"\null" argument as the tag.  Longer comments may
be placed within a \verb"\tablecomment{TEXT}" command.
Lists of table references should use the following
format: \verb"\tablecomment{References:  Names of your references.}".
Only one paragraph of material is permitted  at the end of a table,
(excluding superscripted footnotes), so if both references and notes
exist, they should be run in together.  

Both \verb"\tablenotetext" and \verb"\tablecomment" 
commands must appear after the column heading information
and before the table data; otherwise, they will affect
indentation of the last table cell.  These commands 
work with planotables and tabular tables.


\begin{acknowledgments}
Much of this material has been derived from the work of Patrick W. Daly
for the European Geophysical Society and his efforts in providing
robust and easy to use features are to be commended.  The American
Geophysical Union is also to be thanked for permission to use the
\verb"planotable"  material
\end{acknowledgments}

\begin{thebibliography}{}  
  \bibitem[Butcher(1992)]{bu}  
    Butcher J. 1992. {\it Copy-editing: The Cambridge  
    Handbook}, 3rd edn, Cambridge Univ. Press, Cambridge.
  \bibitem[The Chicago Manual(1982)]{cm}  
    {\it The Chicago Manual of Style}, Univ.   
    Chicago Press, Chicago, 1982.
  \bibitem[Chao(1985)]{ch} 
    Chao, B. F., 1985. Normal mode study of the Earth's rigid 
     body motions, {\it Geophys. Res. Lett.}, {\bf 12}, 526-529.
  \bibitem[Hinderer(1986)]{hi}  
    Hinderer, J., 1986. Resonance effects of the earth's fluid
    core in earth rotation, in {\it Solved and Unsolved 
    Problems}, pp. 277-296, ed. Cazenave A., Reidel, 
    Dordrecht.
  \bibitem[Kopka \& Daly(1995)]{kd}  
    Kopka H. \& Daly P.W., 1995, \textit{A guide to} \LaTeX2e,
    Addison--Wesley, New York  
  \bibitem[Lamport(1986)]{la}  
    Lamport L., 1986,  \LaTeX: {\it A Document   
    Preparation System}, Addison--Wesley, New York  
  \bibitem[Lindberg (1986)]{bl}  
    Lindberg, C., 1986.  Multiple taper harmonic analysis of 
    terrestrial free oscillations, {\it PhD thesis}, 
    University of California.
  \bibitem[Maupin(1992)]{ma} 
    Maupin, V., 1992. Modelling of laterally trapped surface 
    waves with application to Rayleigh waves in the Hawaiian 
    swell, {\it Geophys. J. Int.}. {\bf 110}, 553-570.     	
  \bibitem[Rutherford \& Hawker(1981)]{rh}
      Rutherford, S. R. \& Hawker, K. E., 1981, 
    Consistent coupled mode theory of sound propagation for a 
    class of non-separable problems, 
   {\it J. acoust. Soc. Am.}, {\bf 71}, 554-564
\end{thebibliography}  
  
\end{document}

